\documentclass[12pt,a4paper]{article}

%Pacotes
\usepackage[brazil]{babel} 
\usepackage[T1]{fontenc}
\usepackage[utf8]{inputenc}
\usepackage[left=3cm,top=3cm,right=2cm,bottom=2cm]{geometry}

%Informações do documento
\author{Pero Vaz de Caminha}
\date{ ano de 1500}
\title{A Carta}

%Começa o documento
\begin{document}
\maketitle %gera o titulo do artigo

Pardos, nus, sem coisa alguma que lhes cobrisse suas vergonhas. Traziam
arcos nas mãos, e suas setas. Vinham todos rijamente em direção ao batel. E
{\Huge Nicolau Coelho} lhes fez sinal que pousassem os arcos. E eles os depuseram. Mas
não pôde deles haver fala nem entendimento que aproveitasse, por o mar quebrar
na costa. Somente arremessou-lhe um {\bf barrete vermelho} e uma carapuça de linho
que levava na cabeça, e um \textit{sombreiro preto}. E um deles lhe arremessou um
sombreiro de penas de ave, compridas, com uma copazinha de penas vermelhas e
pardas, como de {\it papagaio}. E outro lhe deu um ramal grande de continhas brancas,
miúdas que querem parecer de aljôfar, as quais peças creio que o Capitão manda a
Vossa Alteza. E com isto se volveu às naus por ser tarde e não poder haver deles
mais fala, por causa do mar.

À noite seguinte ventou tanto sueste com chuvaceiros que fez caçar as naus.
E especialmente a Capitaina. E sexta pela manhã, às oito horas, pouco mais ou
menos, por conselho dos pilotos, mandou o Capitão levantar ancoras e fazer vela. E
fomos de longo da costa, com os batéis e esquifes amarrados na popa, em direção
norte, para ver se achávamos alguma abrigada e bom pouso, onde nós ficássemos,
para tomar água e lenha. Não por nos já minguar, mas por nos prevenirmos aqui. E
quando fizemos vela estariam já na praia assentados perto do rio obra de sessenta
ou setenta homens que se haviam juntado ali aos poucos. Fomos ao longo, e
mandou o Capitão aos navios pequenos que fossem mais chegados à terra e, se
achassem pouso seguro para as naus, que amainassem.

E velejando nós pela costa, na distância de dez léguas do sítio onde tínhamos
levantado ferro, acharam os ditos navios pequenos um recife com um porto dentro,
muito bom e muito seguro, com uma mui larga entrada. E meteram-se dentro e
amainaram. E as naus foram-se chegando, atrás deles. E um pouco antes de solpôsto amainaram também, talvez a uma légua do recife, e ancoraram a onze braças.
E estando Afonso Lopez, nosso piloto, em um daqueles navios pequenos, foi,
por mandado do Capitão, por ser homem vivo e destro para isso, meter-se logo no
esquife a sondar o porto dentro. E tomou dois daqueles homens da terra que
estavam numa almadia: mancebos e de bons corpos. Um deles trazia um arco, e
seis ou sete setas. E na praia andavam muitos com seus arcos e setas; mas não os
aproveitou. Logo, já de noite, levou-os à Capitaina, onde foram recebidos com muito
prazer e festa.

A feição deles é serem pardos, um tanto avermelhados, de bons rostos e
bons narizes, bem feitos. Andam nus, sem cobertura alguma. Nem fazem mais caso
de encobrir ou deixa de encobrir suas vergonhas do que de mostrar a cara. Acerca
disso são de grande inocência. Ambos traziam o beiço de baixo furado e metido nele
um osso verdadeiro, de comprimento de uma mão travessa, e da grossura de um
fuso de algodão, agudo na ponta como um furador. Metem-nos pela parte de dentro
do beiço; e a parte que lhes fica entre o beiço e os dentes é feita a modo de roque
de xadrez. E trazem-no ali encaixado de sorte que não os magoa, nem lhes põe
estorvo no falar, nem no comer e beber.

Os cabelos deles são corredios. E andavam tosquiados, de tosquia alta antes
do que sobre-pente, de boa grandeza, rapados todavia por cima das orelhas. E um
deles trazia por baixo da solapa, de fonte a fonte, na parte detrás, uma espécie de
cabeleira, de penas de ave amarela, que seria do comprimento de um coto, mui
basta e mui cerrada, que lhe cobria o toutiço e as orelhas. 
\end{document}
